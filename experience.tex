\documentclass[12pt,twocolumn]{article}

\begin{document}

\title{A first experience with glad and glfw}

\author{John D.H. Pritchard \thanks{jdp@syntelos.com}}

\date{\today}

\maketitle

%\tableofcontents

\begin{abstract}

The {\tt glfw} looks nice, and it likes the {\tt glad}, so here we go.

Here's some notes on the process of installing and using {\tt glad}
and {\tt glfw}.

\end{abstract}


\section{Installing {\tt glad}}

From {\tt github.com/ddav1dde/glad} the usual clone enters into python
land.  The installation is a python program that looks into and
characterizes the GL environment into its {\it even more portable (?)
}  way.  

As reviewed in {\tt ./INSTALL-GLAD.txt}, the process is friendly to a
bit of experiementation as a regular, non-root, user.  And then it has
need of some archiving.

\section{Installing {\tt glfw}}

Downloading the ZIP from {\tt www.glfw.org} gives you something
usable.  The source repo is in the midst of things undesireable to
someone else's workstation.  This {\tt configure}, {\tt make}, {\tt
  make install} process is simple and straightforward.  

The {\tt glfw} installation even produces a {\tt /usr/lib/pkgconfig}
file.  However, the dependencies don't sort in {\tt pkg-config --libs}
space.  (See {\tt Makefile}, {\it extrapolate as necessary}).

\section{Hello Window}

The {\tt main.cc} program is trivial.  It opens a window.

The {\tt Makefile} includes libraries found via {\tt
  github.com/derhass/HelloCube/blob/master/Makefile}.


\section{Conclusion}

Fun, but a bit time consuming.

%\appendix

%\section{Notes}



\end{document}
